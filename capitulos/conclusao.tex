\mychapter{\textsc{Conclusões}}{chp:conclusoes}
\lhead{\textsc{Conclusões}}

\lettrine{A} seguir apresentamos nossas considerações finais, buscando a contextualização do problema de pesquisa em tela com os resultados obtidos e as sugestões para trabalhos futuros.

\section{\textbf{Considerações Finais}} 

Com a proposta apresentada, buscamos destacar a importância do estudo de redes de coautorias relacionando ao arcabouço teórico de redes complexas e a relação com a colaboração científica, neste trabalho a compreensão da interação existente entre as Universidades Federais do Brasil e a Universidade Federal de Alagoas.

Conforme \citep{Barabasi2001}, os estudos de redes de coautoria são importantes e possuem sua relevância para apresentar padrões de comportamento e possibilitar modelos preditivos de quais áreas ou campos estão se desenvolvendo, são emergentes, ou quais tendências possuem.

O estudo das redes de coautoria deste trabalho, mostrou um resultado que se aplicado a outras redes poderão descrever o comportamento, dinâmica e características, quando se observa as relações existentes dentro das áreas de \textit{Health Sciences, Agricultural Sciences e Exact and Earth Sciences}, a partir da consideração de uma base de indexação de artigos específica ou de outra.

As medidas de centralidade possibilitam aferir conhecimento a respeito da dinâmica e do comportamento das redes de coautorias no aspecto temporal. A aplicação destas, em outras áreas do conhecimento e em outras bases de indexação de artigos científicos, mostram-se significativas para a obtenção de resultados específico para avaliação da colaboração científica, possibilitando a compreensão e caracterização dos atores de relevância e de influência da rede.


\section{\textbf{Trabalhos futuros}}

Por trabalhos futuros, consideramos a expansão do uso de outras medidas de centralidade de rede para análises topológica da interação dos vértices e arestas nas redes de coautoria. Sugerimos outras abordagens de uma prisma analítico e probabilístico, como a utilização de técnicas de \textit{network-driven approaches}, de \textit{link prediction}, análises de comunidades de influências na rede, e simulação do comportamento da rede com base em métricas de redes complexas.



 